\documentclass[letter,11pt,oneside]{article}

%%% (occur "\\(\\\\[a-z]*section\\|appendix\\|input\\|\\<include\\>\\)")

%%\documentclass[11pt,twocolumn]{article}
%%\usepackage[inline]{asymptote}   %% Inline asymptote diagrams
%%\usepackage{wglatex}             %% Use this one and kill others.
\usepackage{color}               %% colored letters {\color{red}{{text}}
\usepackage{fancyhdr}            %% headers/footers
%%\usepackage{fancyvrb}            %% headers/footers
\usepackage{datetime}            %% pick up tex date time 
\usepackage{lastpage}            %% support page of ...lastpage
\usepackage{times}               %% native times roman fonts
\usepackage{textcomp}            %% trademark
\usepackage{amssymb,amsmath}     %% greek alphabet
\usepackage{parskip}             %% blank lines between paragraphs, no indent
\usepackage{shortvrb}            %% short verb use for tables
\usepackage{lscape}              %% landscape for tables.
\usepackage{longtable}           %% permit tables to span pages wg-longtable
\usepackage{url}                 %% Make URLs uniform and links in PDFs
\usepackage{enumerate}           %% Allow letters/decorations for enumerations
\usepackage{endnotes}            %% Enhance footnotes/endnotes
\usepackage{listings}            %% Make URLs uniform and links in PDFs
\pdfadjustspacing=1                %% force LaTeX-like character spacing
\usepackage{geometry}            %% allow margins to be relaxed
%%\usepackage{wrapfig}             %% permit wrapping figures.
%%\usepackage{subfigure}              %% images side by side.
\geometry{margin=1in}            %% Allow narrower margins etc.
\usepackage[T1]{fontenc}         %% Better Verbatim Font.
\renewcommand*\ttdefault{txtt}   %% 
%%\usepackage{natbib}   %% bibitems

%% include background image (wg-document-page-background) 

\usepackage{graphicx}            %% Include pictures into a document
%% (wg-texdoc-inserttikz)


\def\documentisdraft{NOTDRAFT}

%% (wg-texdoc-isdraft)
%% (wg-texdoc-insert-fancy-headers)

%%\usepackage[bookmarks]{hyperref} %% Make huperlinks within a PDF
%%\usepackage{makeidx}             %% Make an index uncomment following line
%%\makeindex                       %%.. yeah this one, too. index{key} in text
%%



\definecolor{verbcolor}{rgb}{0.6,0,0}
\definecolor{darkgreen}{rgb}{0,0.4,0}
\newcommand\debate[1]{\textcolor{darkgreen}{DEBATE: #1} \marginpar{\textcolor{red}{DEBATE} }}
\newcommand{\ltodo}[2]{\marginpar{\textcolor{red}{ACTION: #1}\endnote{#2}}}
\renewcommand{\thefigure}{\thesection-\arabic{figure}}
\newcommand{\menu}{\ensuremath{\;\rightarrow\;}}
%%(wg-add-inline-images)  %% add inline images to the mix





%%Begin User Definitions: Hint: ~/.latex.defs and  latex.defs  
%%End User Definitions:
%%(wg-texdoc-adjust-paper-width)
%% (wg-texdoc-insert-hypersetup)



%%%%%%%%%%%%%%%%%%%%%%%%%%%%%%%%%%%%%%%%%%%%%%%%%%%%%%%%%%%%%%%%%%%%%%%%%%%%%


\begin{document}


%% (wg-latex-pretty-title-page)
%% (wg-texdoc-titleblock)

\setcounter{section}{0}
\pagenumbering{arabic}

\ifx\documentisdraft\drafttest
\linenumbers    %%%%%%%%%%%%% DRAFT
\fi

\section{Using GitHub SAS Spectroscopy Repository}

Github\texttrademark is a system used by businesses and
opensource projects to promote team collaboration. The
tools and capabilities are far in excess to what we need,
but the cost is free. Hard to beat.

The core of Github is the file revision management system. This
is the same system that manages the code for the Linux kernel.
It was designed by top-level programmers for programmers. It is
NOT Dropbox\texttrademark. 

Woody and I have been using its Atlassian analog, ``Bitbucket''\texttrademark
by virtue of an account I had there. It provides a way we can
synchronize his data with my explorations in data reduction.

I find Github to be easier to use, and we settled on that platform
for the SAS Spectroscopy work.

First some basics:

I installed the git tools on my Linux system and use it with
a command line interface. Much faster, more flexible, and --
well easier for me.

Windows users can download the Git GUI and use the system in that fashion.



\section{Overview}

The SAS Spectroscopy repository has a README.md file. This is the ``new''
way of producing nice content using a markdown language. It is prett
simple and weak it the same time.

The SAS Spectroscopy repository has an associated Wiki.

To collaborate fully in this setting, you will need to establish
an account with Github. With your Github credentials, you can
be added to the mix.





\section{Forum Like Interaction}

The dashboard for the SAS Spectroscopy repository has a tab
labeled ``projects''. Under that heading I have added:

\begin{table}[h!]
%\phantomsection
%\addcontentsline{toc}{section}{ TOC CAPTION}
% \setlength{\belowcaptionskip}{6pt} % adjust space under caption abovecaptionskip
% \renewcommand{\arraystretch}{1.3} % adjust line spacing
%\small{
%\begin{minipage}{\textwidth}     % for footnotes in table.
%\caption[TOC]{ProjectTable}
\centering
\begin{tabular}{ l  l }
%\MakeShortVerb{\|}
%\multicolumn{n}{fmt}{text for merged cols}
Topic  & Relevance   \\
Questions  & -- for general questions etc.    \\ 
Hardware   & -- for hardware related questions    \\ 
Software   & -- For sofware related questions    \\ 
Automation & -- questions related to automation.    \\ 
%% ones-based: \cline{a-b}

%%\DeleteShortVerb{|}
\end{tabular}
%%\end{minipage}    %% for footnotes  r@{.}l 
\caption{Projects currently available}
\label{table:ProjectTable}
%%} % end small etc
\end{table}



%%\appendix
%%\renewcommand \thesection{\Alph{section}}

%% use a bibitem approach to the references publications etc.
%% (wg-bibitem)

%%\clearpage
%%\addcontentsline{toc}{section}{References}
%%\renewcommand*{\refname}{My Bibliography and References}
%%\bibliographystyle{plain}	% bibliographystyle{apalike} and \usepackage{natbib}
%%\bibliography{MasterBib}	% expects file "MasterBib.bib"


%%\clearpage
%%\addcontentsline{toc}{section}{Index}
%%\printindex %% www.cs.usask.ca/resources/tutorials/latex/notes/toc-index.pdf

%%\begin{thebibliography}{80}
%%\usepackage{natbib}   %% bibitems
%%\end{thebibliography}

% /home/wayne/git/SAS_Spectroscopy/doc/GitHub.tex

%% (wg-texdoc-endnotes)
\end{document}
